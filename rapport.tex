\documentclass[11pt,a4paper]{article}

\usepackage[utf8]{inputenc}
\usepackage[francais]{babel}
\usepackage[T1]{fontenc}

\usepackage{graphicx}
\usepackage[margin=3cm]{geometry}
\usepackage{hyperref} %Pour les liens dans le sommaire
\hypersetup{                    % parametrage des hyperliens
    colorlinks=true,                % colorise les liens
    breaklinks=true,                % permet les retours à la ligne pour les liens trop longs
    urlcolor= black,                 % couleur des hyperliens
    linkcolor= black,                % couleur des liens internes aux documents (index, figures, tableaux, equations,...)
    citecolor= black                % couleur des liens vers les references bibliographiques
    }
\usepackage{lscape} %Pour le format paysage
\usepackage{fancyhdr} %Pour le format du rapport
\usepackage{xcolor} %Pour l'utilisation de couleur dans le texte
\usepackage{tabularx}
\frenchbsetup{StandardLists=true}
\usepackage{enumitem}
\usepackage{pifont}
\usepackage{listings}
\usepackage{minted}


\usepackage{amsmath}
\usepackage{amsfonts}
\usepackage{amssymb}
\usepackage{listings}

\def\code#1{\texttt{#1}} %Pour insérer du code sur une ligne


\definecolor{myGreen}{rgb}{0,0.46,0.38} %#007561
\definecolor{myKaki}{rgb}{0.60,0.45,0} %#997200
\definecolor{myBlue}{rgb}{0,0.45,0.68} %#0073ad
\definecolor{myPurple}{rgb}{0.51,0.25,0.76} %#8241c2
\definecolor{myOrange}{rgb}{0.95,0.50,0.05} %#f17f0d
\definecolor{myDarkBlue}{rgb}{0,0,0.63} %#0000a0
\definecolor{myDarkRed}{rgb}{0.44,0,0} %#700000
\definecolor{myRed}{rgb}{0.73,0.04,0.11} %e00909
\definecolor{myGrey}{rgb}{0.439,0.439,0.439} %707070


\author{Vincent BEUGNET - Adlane LADJAL}
\title{Devoir de Programmation Fonctionnelle \\ Opérations sur des grammaires hors-contexte}


\begin{document}

\lstset{language=caml,
 columns=[c]fixed,
 basicstyle=\small\ttfamily,
 keywordstyle=\bfseries,
 upquote=true,
 commentstyle=,
 breaklines=true,
 showstringspaces=false}          % Set your language (you can change the language for each code-block optionally)

\makeatletter
\begin{titlepage}
	\centering
	\includegraphics[width=0.25\textwidth]{logo_supgalilee.jpg}
	\hfill
	\includegraphics[width=0.25\textwidth]{logo-paris13_bis.png} \\
    \vspace{5cm}
       {\LARGE \textbf{\@title}} \\
    \vspace{2em}
        {\large \@author }\\
    \vspace{1em}
        {\textit{\@date}} \\
    \vspace{2em}
    	\vspace{2em}
    		{Enseignant : Christian CODOGNET} \\
    \vfill
\end{titlepage}


%insertion de la page blanche
\newpage
~
\newpage

%Table des matieres
\renewcommand{\contentsname}{Sommaire}
\tableofcontents

\newpage

\pagestyle{plain}

\section{Introduction}
Le présent devoir a été réalisé par Vincent Beugnet
et Adlane Ladjal, au cours de notre deuxième année
de formation d'Ingénieur Informatique à  Sup Galilée.
Il a été réalisé dans le cadre du cours Programmation
Fonctionnelle. Le but de ce cours est de nous donner
les bases de ce paradigme de programmation, en nous
initiant au lambda-calcul, ainsi qu'aux algorithmes de
typage et d'unification.

Nous devions concevoir un programme OCAML permettant de
réaliser trois opérations sur des grammaires hors-contextes :

\begin{itemize}
    \item Déterminer les non-terminaux accessibles.
    \item Déterminer les non-terminaux productibles.
    \item Eliminer les epsilon-règles.
\end{itemize}

Une grammaire hors-contexte est une grammaire où toutes les
règles de production sont de la forme \[ X \rightarrow \alpha \]
où $X$ est un symbole non terminal, et $\alpha$ un mot composé
de symboles non terminaux et/ou terminaux. $\alpha$ peut aussi
être le mot vide $\epsilon$.
\newpage


\section{Définition des types OCAML}

Nous devions commencer tout d'abord par définir les types OCAML.
\newline

\subsection{Le type \code{lettre}}
Tout d'abord nous avons définis un type \code{lettre}. Ce type
représente à la fois les non terminaux, les terminaux et $\epsilon$.

Nous aurions pu représenter les non terminaux, les terminaux et
Epsilon séparemment. Il nous aurait aussi eu fallu le type \code{lettre}
pour pouvoir avoir une liste de \code{lettre} qui correspond à $\alpha$
dans une production $X \rightarrow \alpha$.

Dans ce cas-là nous aurions alors quelque chose comme cela :
\begin{minted}{ocaml}
type nonTerminal = NT of char ;;
type terminal = T of char ;;
type epsilon = Epsilon ;;
type lettre = terminal | nonTerminal | epsilon ;;
\end{minted}

Mais une telle définition de \code{lettre} n'est pas possible.

Dès lors nous avons opté pour cette définition :
\begin{minted}{ocaml}
type lettre = T of char | NT of char | Epsilon ;;
\end{minted}

Ici, nous pouvons remarquer que pour les non-terminaux,
par exemple, rien ne nous empêche de les représenter par
une lettre minuscule, par exemple : \code{NT('a')}.
Alors que d'usage,
nous les représentons par une lettre majuscule.
Il en va de même pour les terminaux avec \code{T('A') } par
exemple.

Nous n'avons pas considérer ceci comme un problème. Si
l'utilisateur utilise \code{NT('A')} et \code{NT('a')}, le
programme interprétera ces deux symboles comme différents.

Nous, nous ferons attention à bien représenter les non terminaux
par des majuscules et les terminaux par des minuscules.

\subsection{Le type \code{regle}}

Nous avons ensuite définis le type des règles de production.
Nous avons considéré ce type comme composé de deux champs, le
terme de gauche, un symbole non terminal, et le terme de droite
une liste de symboles non terminaux et terminaux, ou une liste
contenant simplement \code{Epsilon}.

\begin{minted}{ocaml}
type regle = Prod of (lettre * (lettre list)) ;;
\end{minted}

Nous remarquons qu'avec cette définition, nous pouvons
avoir un symbole terminal à gauche, ce qui ne correspond pas
à une grammaire hors-contexte.

En revanche notre programme n'accepte en entrée uniquement des
grammaires hors-contextes. Nous pouvons garantir le bon fonctionnement
du programme uniquement pour des grammaires hors-contextes.
Dès lors, si l'utilisateur rentre en entrée une grammaire qui
n'est pas hors-contexte, le programme ne pourra fonctionner
correctement. Il appartient à l'utilisateur de prendre à rentrer
une grammaire hors-contexte.

\newpage

%%%%%%%%%%%%%%%%%%%%%%%%%%%%%%%%%%%%%%%%%%%%%%%%%%%%%%%%%

\section{Les non terminaux accessibles}

On note $A \rightarrow^* \alpha$ le fait que $\alpha$ est atteint après
0 ou plus dérivation(s) à partir de $A$. Une dérivation est l'application d'une règle de la grammaire.

Un non terminal accessible $B$ à partir d'un non terminal $A$ est tel que
$B \in \alpha$ avec $A \rightarrow^* \alpha$
\newline

La première opération de notre programme est donc de récupérer les
non-terminaux accessibles à partir d'un non terminal donné
en entrée. Notre fonction récursive OCAML prendra alors en
entrée deux paramètres :
\begin{itemize}
    \item Une grammaire sous forme d'une liste de productions.
    \item Un symbole non terminal.
\end{itemize}

Nous avons décomposé cette fonction en deux étapes. Nous avons une
fonction qui exécute la tâche principale décrite plus haut. Et une
autre fonction qui se charge de récupérer tous les non terminaux
qui sont produits par un symbole après une seule dérivation.

L'opération de récupération des non terminaux accessibles se réalise
comme suit :

\begin{itemize}
    \item La fonction récursive est initialisée avec une liste
    de non terminal \code{parcours} qui contient uniquement
    dans un premier lieu le non terminal de départ.
    Et une liste de tous les non terminaux que contient
    la grammaire. Nous l'appelons \code{alphabet}.
    \item \code{parcours} est parcouru. Pour chaque non terminal $T$
    de \code{parcours}, nous récupérons tous les non terminaux
    que peux produire $T$ avec une seule dérivation. Ceci nous fournit
    une liste de non terminaux \code{acc}.
    \item Ensuite nous concaténons cette liste dans le résultat
    retourné par la fonction. Nous concaténons aussi \code{acc}
    à \code{parcours} pour pouvoir calculer les non terminaux
    accessibles directement de ces nouveaux symboles.
    Puis nous supprimons le non terminal $T$ de \code{alphabet}
    dont nous venons de réaliser le calcul pour éviter de boucler indéfiniment.
\end{itemize}

Le code est fourni en annexe.

\newpage

\section{Les non terminaux productibles}

\newpage

\section{La suppression des $\epsilon$-règles}

Les $\epsilon$-règles sont des règles de la forme
$A \rightarrow \epsilon$ où $A$ est un symbole non
terminal, et $\epsilon$ le mot vide.

Le but de l'opération de suppression des $\epsilon$-règles
est de passer d'une grammaire hors-contexte $G$ quelconque
à une grammaire $\epsilon$-libre $G^\prime$, c'est-à-dire ne
contenant pas d'$\epsilon$-règles.

La grammaire $G^\prime$ devra dériver exactement le même
langage que la grammaire $G$. Seule différence : la grammaire
$G^\prime$ ne pourra générer le mot vide.
\newline

Notre fonction récursive OCAML prendra alors en paramètre
la grammaire hors-contexte que l'on souhaite transformer
en une grammaire $\epsilon$-libre.

Elle se déroule comme suit :

\begin{itemize}
    \item En premier lieu nous récupérons dans la liste \code{ntpe}
    les symboles non terminaux qui produisent le mot vide.
    \item Nous parcourons la liste \code{ntpe}. Puis pour chacun de
    ces symboles nous supprimons les $\epsilon$-règles associées au symbole.
\end{itemize}

Lorsque nous supprimons les $\epsilon$-règles associées à un
symbole T, nous devons distinguer deux cas :
\begin{description}
    \item[Cas 1] Le cas où seuls des règles de la forme
        $T \rightarrow \epsilon$ et $T \rightarrow \theta$
        (avec $\theta$ contenant un nombre indéfini de
        fois le symbole $T$ uniquement) apparaissent. Dans ce cas, il faut
        tout simplement supprimer le symbole $T$ de la grammaire et ne réaliser
        aucune autre opération. En effet, $T$ ne produit rien, et est
        donc inutile à la grammaire.
    \item[Cas 2] Le cas où la règle $T \rightarrow \epsilon$ apparaît avec d'autres
        règles de la forme $T \rightarrow \alpha$, $\alpha$ contenant
        à la fois des non terminaux et des terminaux. Dans ce cas nous supprimons
        la règle $T \rightarrow \epsilon$, et pour toutes les autres règles de la grammaire
        contenant le symbole $T$ dans le membre de droite nous appliquons cette
        procédure :

        Si nous avons une règle de la forme $A \rightarrow \alpha T \beta$,
        avec $\alpha$ et $\beta$ contenant à la fois des non terminaux et des terminaux
        (ils peuvent aussi produire le symbole T), alors il faut rajouter la règle
        $A \rightarrow \alpha \beta$ dans la grammaire.
\end{description}

Dans le cas 2, pour rajouter les règles lorsque l'on trouve le symbole à traiter dans
le membre de droite d'une production, nous procédons de la sorte :

Nous générons une liste des positions dans le membre de droite que prend le symbole à traiter.
Par exemple dans la règle $S \rightarrow aSbScS$, nous aurons la liste suivante :

\code{[1;3;5]}

Puis à partir de cette liste nous générons une liste de toutes les combinaisons possibles.

Avec l'exemple nous aurons : \code{[[1]; [3]; [5]; [1; 3]; [1; 5]; [3; 5]; [1; 3; 5]]}

Puis nous parcourons cette dernière liste et nous ajoutons les règles sans les symboles
non terminaux aux positions données.

Donc toujours avec le même exemple, lorsque nous tombons sur la liste \code{[1; 5]} à traiter, il faut ajouter
la règle $S \rightarrow abSc $.
\newpage

\section{Conclusion}
Ce devoir nous a permis de nous familiariser un peu plus avec la
programmation fonctionnelle. Il nous a permis d'utiliser les notions que nous
avons acquises en Théorie des Langages, mais aussi en Compilation pour les appliquer
dans ce projet.


\end{document}
